% -----------------------------------------------------------------------------
% Copyright &copy; 2015 Ben Blazak <bblazak@fullerton.edu>  
% This work is licensed under a [Creative Commons Attribution 4.0 International
% License] (http://creativecommons.org/licenses/by/4.0/)  
% -----------------------------------------------------------------------------

% -----------------------------------------------------------------------------
% Copyright &copy; 2015 Ben Blazak <bblazak@fullerton.edu>  
% This work is licensed under a [Creative Commons Attribution 4.0 International
% License] (http://creativecommons.org/licenses/by/4.0/)  
% -----------------------------------------------------------------------------

% - [Which lua environment should I use?] (http://tex.stackexchange.com/a/33102)
\directlua{require('scripts/functions.lua')}

\ifx\question\undefined
  \long\def \question #1{#1}
\fi
\ifx\answer\undefined
  \long\def \answer #1{#1}
\fi

% -----------------------------------------------------------------------------

\def \docTitle  {\directlua{doc_title('\jobname')}}
\def \docAuthor {\question{Name:\hfill{}CWID:\hfill}\answer{Answer Key}}
\def \docClass  {CPSC 121-11}
\def \docSchool {~}
\def \docTerm   {CSUF Fall 2015}

\def \docCopyright {
  \begin{figure}[b!]
    \tiny
    {\answer{\color{red}}\question{\color{.}}\hrule}
    \vspace{2ex}
    \begin{minipage}[c]{0.10\textwidth} \vspace{0ex} \vspace{-1pt}
      \includegraphics[height=5ex]{images/ccby}
    \end{minipage}
    \begin{minipage}[c]{0.90\textwidth} \vspace{0ex}
      Copyright \copyright{} Ben Blazak \url{bblazak@fullerton.edu}\\
      This work is licensed under a Creative Commons Attribution 4.0
      International License \url{http://creativecommons.org/licenses/by/4.0/}
    \end{minipage}
    \vspace{-.14in}
    \vspace{2ex}
  \end{figure}
}

% -----------------------------------------------------------------------------

\documentclass{article}

\usepackage[includehead,
%             includefoot,
            margin=1in,
            top=.25in,
            headheight=.75in,
            headsep=.25in,
            footskip=.25in,
           ]{geometry}

\usepackage[fleqn]{amsmath}
\usepackage{amssymb}
\usepackage[shortlabels]{enumitem}
\usepackage{fancyhdr}
\usepackage{lastpage}
\usepackage{minted}
\usepackage{tikz}
\usepackage[explicit]{titlesec}

\usepackage[T1]{fontenc}
\usepackage[utf8]{inputenc}
\usepackage{lmodern}

\usepackage{xcolor}
\usepackage[framemethod=TikZ]{mdframed}

% - [Illegal parameter number in definition of \Hy@tempa]
%   (https://kba49.wordpress.com/2013/04/12/illegal-parameter-number-in-definition-of-hytempa/)
\usepackage{hyperref}

% text ------------------------------------------------------------------------

\binoppenalty = 10000  % never break next to a binary operator
\relpenalty   = 10000  % never break next to a relation operator

\setlength{\parindent}{0em}
\setlength{\parskip}{1ex}

\setlist[itemize]{nosep,itemsep=.5ex,parsep=.5ex}

% sections --------------------------------------------------------------------

% - <http://tex.stackexchange.com/a/58299>
\titleformat{\section}{\normalfont\Large\bfseries}{}{0em}{#1\ \thesection}

% math ------------------------------------------------------------------------

\setlength{\mathindent}{1cm}

% - "\begin{document}" resets these values, so they have to be treated
%   specially
\AtBeginDocument{
  \setlength{\abovedisplayskip}{1.5ex plus .5ex minus .5ex}
  \setlength{\belowdisplayskip}{1.5ex plus .5ex minus .5ex}
}

% source code -----------------------------------------------------------------

\usemintedstyle{solarizedlight}

\fvset{samepage=true}

% header and footer -----------------------------------------------------------

\pagestyle{fancy}

\renewcommand{\headrule}{{\answer{\color{red}}\question{\color{.}}
                          \vskip 0.5ex \hrule height 0.4pt \vskip 0.5ex}}
\renewcommand{\footrule}{{\answer{\color{red}}\question{\color{.}}
                          \vskip 0.5ex \hrule height 0.4pt \vskip 0.5ex}}

\fancyhead[L]{{\answer{\color{red}}\question{\color{.}}\docAuthor}
              {\color{.}\\\docClass}}
\fancyhead[C]{{\color{.}\docSchool\\}}
\fancyhead[R]{{\color{.}\docTerm\\\docTitle}}

\fancyfoot[C]{\thepage{} of \pageref{LastPage}}

% -----------------------------------------------------------------------------
% macros
% -----------------------------------------------------------------------------

% abbreviations ---------------------------------------------------------------

\def \<{\langle}
\def \>{\rangle}

\def \ε{\varepisilon}
\def \θ{\vartheta}
\def \κ{\varkappa}
\def \π{\varpi}
\def \ρ{\varrho}
\def \σ{\varsigma}
\def \φ{\varphi}

\def \Γ{\varGamma}
\def \Δ{\varDelta}
\def \Θ{\varTheta}
\def \Λ{\varLambda}
\def \Ξ{\varXi}
\def \Π{\varPi}
\def \Σ{\varSigma}
\def \Υ{\varUpsilon}
\def \Φ{\varPhi}
\def \Ψ{\varPsi}
\def \Ω{\varOmega}

% special characters ----------------------------------------------------------

\catcode `α = \active \let α \alpha
\catcode `β = \active \let β \beta
\catcode `γ = \active \let γ \gamma
\catcode `δ = \active \let δ \delta
\catcode `ε = \active \let ε \epsilon
\catcode `ζ = \active \let ζ \zeta
\catcode `η = \active \let η \eta
\catcode `θ = \active \let θ \theta
\catcode `ι = \active \let ι \iota
\catcode `κ = \active \let κ \kappa
\catcode `λ = \active \let λ \lambda
\catcode `μ = \active \let μ \mu
\catcode `ν = \active \let ν \nu
\catcode `ξ = \active \let ξ \xi
\catcode `ο = \active \let ο o
\catcode `π = \active \let π \pi
\catcode `ρ = \active \let ρ \rho
\catcode `σ = \active \let σ \sigma
\catcode `τ = \active \let τ \tau
\catcode `υ = \active \let υ \upsilon
\catcode `φ = \active \let φ \phi
\catcode `χ = \active \let χ \chi
\catcode `ψ = \active \let ψ \psi
\catcode `ω = \active \let ω \omega

\catcode `Α = \active \let Α A
\catcode `Β = \active \let Β B
\catcode `Γ = \active \let Γ \Gamma
\catcode `Δ = \active \let Δ \Delta
\catcode `Ε = \active \let Ε E
\catcode `Ζ = \active \let Ζ Z
\catcode `Η = \active \let Η H
\catcode `Θ = \active \let Θ \Theta
\catcode `Ι = \active \let Ι I
\catcode `Κ = \active \let Κ K
\catcode `Λ = \active \let Λ \Lambda
\catcode `Μ = \active \let Μ M
\catcode `Ν = \active \let Ν N
\catcode `Ξ = \active \let Ξ \Xi
\catcode `Ο = \active \let Ο O
\catcode `Π = \active \let Π \Pi
\catcode `Ρ = \active \let Ρ P
\catcode `Σ = \active \let Σ \Sigma
\catcode `Τ = \active \let Τ T
\catcode `Υ = \active \let Υ \Upsilon
\catcode `Φ = \active \let Φ \Phi
\catcode `Χ = \active \let Χ X
\catcode `Ψ = \active \let Ψ \Psi
\catcode `Ω = \active \let Ω \Omega

% other -----------------------------------------------------------------------

\long\def \textAnswer #1{\answer{{\color{red!50!black}#1}}}
\long\def \textQuestion #1{\question{{\color{.}#1}}}

\long\def \standardsSection #1{{
  \section*{Standards}
  \def\arraystretch{2}
  \setlength\tabcolsep{1em}
  \begin{tabular}{|r|l|}
    \hline \textbf{Standard} & \textbf{Score} \\\hline #1
  \end{tabular}
  \docCopyright
  \newpage
}
  \def\docCopyright{}
}
\long\def \standards #1{{
  \vspace{-1em}
  \def\arraystretch{2}
  \setlength\tabcolsep{1em}
  \begin{tabular}{|r|l|}
    \hline #1
  \end{tabular}
  \vspace{1em}
}}
\def \standard #1{
  #1 & ~~~ \\\hline
}

\def \program #1{{
  \def \lCodedir{\directlua{doc_name('\jobname')}}
  \def \lCode{./code/\lCodedir/#1.cpp}
  \def \lQuestion{./code/\lCodedir/#1.gen.question.txt}
  \def \lAnswer{./code/\lCodedir/#1.gen.answer.txt}
  %
  \inputminted[frame=single]{cpp}{\lCode}
  \textQuestion{
    \inputminted[frame=single,
                 label={\normalsize{}Output},
                 fontsize=\Large,
                ]{text}{\lQuestion}
  }
  \textAnswer{
    \inputminted[frame=single,
                 label={Output},
                ]{text}{\lAnswer}
  }
}}



\def \hrulegray{{\color{lightgray}\hrule}}

% -----------------------------------------------------------------------------

\begin{document}
\docCopyright

\section{Question}
\standards{
  \standard{object-oriented programming :: general knowledge}
}

Give brief definitions of the following terms:
\begin{itemize}
    \hrulegray
  \item object
    \textAnswer{\\
      A collection of data (what the object is made of) and methods (or member
      functions) on that data (what the object can do, or what messages the
      object understands) having a type (what the object is seen as).
    }
    \vfill\vfill\hrulegray
  \item class
    \textAnswer{\\
      A description of (or a template (though not in the C++ and Java sense) for
      creating) an object.
    }
    \vfill\hrulegray
  \item instance
    \textAnswer{\\
      A specific object existing in memory.
    }
    \vfill\hrulegray
  \item superclass
    \textAnswer{\\
      A class that is a parent of the class to which we are referring (i.e.~a
      class from which we directly or indirectly inherit).
    }
    \vfill\hrulegray
  \item subclass
    \textAnswer{\\
      A class that is a child of the class to which we are referring (i.e.~a
      class that directly or indirectly inherits from us).
    }
    \vfill\hrulegray
  \item member
    \textAnswer{\\
      A datum or method belonging to a class.
    }
    \vfill\hrulegray
  \item method
    \textAnswer{\\
      Or member function.  An action that an object is able to perform, or a
      message that an object is able to understand.
    }
    \vfill\hrulegray
  \item getters and setters
    \textAnswer{\\
      Methods used for retrieving (getting) or modifying (setting) an object's
      data members.
    }
    \vfill\hrulegray
  \item accessors and mutators
    \textAnswer{\\
      The same as getters and setters.
    }
    \vfill\hrulegray
  \item access modifiers
    \begin{itemize}
      \item public
        \textAnswer{\\
          Visible to everyone.
        }
        \vfill\hrulegray
      \item private
        \textAnswer{\\
          Visible within the class, and to friends.
        }
        \vfill\hrulegray
      \item protected
        \textAnswer{\\
          Visible within the class, to child classes, and to friends.
        }
        \vfill\hrulegray
    \end{itemize}
\end{itemize}

\newpage

\section{Question}
\standards{
  \standard{classes}
  \standard{documentation}
  \standard{consistency of style}
}

Preface your program with a block comment containing a short description of
what it does, and with all necessary \mintinline{cpp}{#include} and
\mintinline{cpp}{using} statements.

Next, write a \mintinline{cpp}{Character} class with the following members:
\begin{itemize}
  \item A \mintinline{cpp}{name}, as a string, with either
    \mintinline{cpp}{private} or \mintinline{cpp}{protected} visibility.
  \item A default constructor that optionally accepts a constant reference to a
    \mintinline{cpp}{string} and initializes the \mintinline{cpp}{name} data
    member to that value.
  \item A \mintinline{cpp}{void} method named \mintinline{cpp}{sayName} that
    prints \mintinline{cpp}{"My name is "} followed by the
    \mintinline{cpp}{Character}'s name, followed by a newline, to standard
    output.
\end{itemize}

Next, write an \mintinline{cpp}{AvatarCharacter} class that inherits from
\mintinline{cpp}{Character} and has the following members:
\begin{itemize}
  \item An \mintinline{cpp}{element}, as a string, with either
    \mintinline{cpp}{private} or \mintinline{cpp}{protected} visibility.
  \item A default constructor that optionally accepts two constant references
    to \mintinline{cpp}{string}s, passes the first value to the parent
    constructor, and uses the second to initialize the
    \mintinline{cpp}{element} data member.
  \item A \mintinline{cpp}{void} function named \mintinline{cpp}{sayElement}
    that sends \mintinline{cpp}{"I bend "}, followed by the
    \mintinline{cpp}{AvatarCharacter}'s \mintinline{cpp}{element}, followed by
    a newline, to standard output.
\end{itemize}

Finally, write a short \mintinline{cpp}{main()} that calls the
\mintinline{cpp}{sayName()} and \mintinline{cpp}{sayElement()} methods on
objects of type \mintinline{cpp}{AvatarCharacter} to produce the following
output:
\inputminted[label=Output]{text}{\docCodeDir/.main.gen.output}

Before you begin, please
\begin{itemize}
  \item Be sure to write your answers neatly and in a good and consistent
    style.  This will be graded.  I recommend writing out your solution on the
    back of the test or on scratch paper and then copying it to page 3 (make
    sure to cross out the version you don't want graded).  You might also
    consider using pencil, if pen is your usual choice.
  \item Try to comment with the level of detail you would find helpful (but not
    irritating) in code of similar complexity written by another student in
    this class.
\end{itemize}

\newpage

\textQuestion{~}
\textAnswer{\inputminted{cpp}{\docCodeDir/main.cpp}}

\end{document}

