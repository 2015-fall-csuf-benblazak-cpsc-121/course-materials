% vim: foldmethod=marker foldmarker={{{{{,}}}}} foldlevel=0
% -----------------------------------------------------------------------------
% Copyright &copy; 2015 Ben Blazak <bblazak@fullerton.edu>  
% This work is licensed under a [Creative Commons Attribution 4.0 International
% License] (http://creativecommons.org/licenses/by/4.0/)  
% -----------------------------------------------------------------------------

\def \docTitle  {Quiz 01}

% -----------------------------------------------------------------------------
% document setup
% -----------------------------------------------------------------------------
% {{{{{

\def \docAuthor {\question{Name:\hfill{}CWID:\hfill{}}\answer{Answer Key}}
\def \docClass  {CPSC 121-11}
\def \docSchool {~}
\def \docTerm   {CSUF Fall 2015}

\def \docCopyright {{
  \tiny
%   \vfill
%   \vspace{\parskip}
%   {\answer{\color{red}}\question{\color{.}}\hrule}
%   \vspace{0\parskip}
  \begin{minipage}[c]{0.10\textwidth} \vspace{0ex}
    \includegraphics[height=5ex]{images/ccby}
  \end{minipage}
  \begin{minipage}[c]{0.90\textwidth} \vspace{0ex}
    Copyright \copyright{} Ben Blazak \url{bblazak@fullerton.edu}\\
    This work is licensed under a Creative Commons Attribution 4.0
    International License \url{http://creativecommons.org/licenses/by/4.0/}
  \end{minipage}
%   \vspace{-2.5\parskip}
}}
\def \docCopyrightTest {{
  \tiny
  \vfill
  \vspace{\parskip}
  {\answer{\color{red}}\question{\color{.}}\hrule}
  \vspace{0\parskip}
  \begin{minipage}[c]{0.10\textwidth} \vspace{0ex}
    \includegraphics[height=5ex]{images/ccby}
  \end{minipage}
  \begin{minipage}[c]{0.90\textwidth} \vspace{0ex}
    Copyright \copyright{} Ben Blazak \url{bblazak@fullerton.edu}\\
    This work is licensed under a Creative Commons Attribution 4.0
    International License \url{http://creativecommons.org/licenses/by/4.0/}
  \end{minipage}
  \vspace{-2.5\parskip}
}}

% -----------------------------------------------------------------------------

\documentclass{article}

\usepackage[includehead,
            includefoot,
            margin=1in,
            top=.25in,
            headheight=.75in,
            headsep=.25in,
            footskip=.25in,
           ]{geometry}

\usepackage[fleqn]{amsmath}
\usepackage{amssymb}
\usepackage[shortlabels]{enumitem}
\usepackage{fancyhdr}
\usepackage{lastpage}
\usepackage{minted}
\usepackage{tikz}
\usepackage{url}

\usepackage[T1]{fontenc}
\usepackage[utf8]{inputenc}
\usepackage{lmodern}

\usepackage{xcolor}
\usepackage[framemethod=TikZ]{mdframed}

% text ------------------------------------------------------------------------

\binoppenalty = 10000  % never break next to a binary operator
\relpenalty   = 10000  % never break next to a relation operator

\setlength{\parindent}{0em}
\setlength{\parskip}{1ex}

\setlist[itemize]{nosep,itemsep=.5ex,parsep=.5ex}

% math ------------------------------------------------------------------------

\setlength{\mathindent}{1cm}

% - "\begin{document}" resets these values, so they have to be treated
%   specially
\AtBeginDocument{
  \setlength{\abovedisplayskip}{1.5ex plus .5ex minus .5ex}
  \setlength{\belowdisplayskip}{1.5ex plus .5ex minus .5ex}
}

% source code -----------------------------------------------------------------

\usemintedstyle{solarizedlight}

\fvset{samepage=true}

% header and footer -----------------------------------------------------------

\pagestyle{fancy}

\renewcommand{\headrule}{{\answer{\color{red}}\question{\color{.}}
                          \vskip 0.5ex \hrule height 0.4pt \vskip 0.5ex}}
\renewcommand{\footrule}{{\answer{\color{red}}\question{\color{.}}
                          \vskip 0.5ex \hrule height 0.4pt \vskip 0.5ex}}

\fancyhead[L]{{\answer{\color{red}}\question{\color{.}}\docAuthor}
              {\color{.}\\\docClass}}
\fancyhead[C]{{\color{.}\docSchool\\}}
\fancyhead[R]{{\color{.}\docTerm\\\docTitle}}

\fancyfoot[C]{\thepage{} of \pageref{LastPage}}

\fancypagestyle{firstpage}{
%   \vskip -4ex \vskip -.8pt  % height of the copyright block
  \renewcommand{\footrule}{{
    \vskip 1ex
    {\answer{\color{red}}\question{\color{.}}
     \vskip 0.5ex \hrule height 0.4pt \vskip 0.5ex}
    \vskip .5ex
    \docCopyright{}
    \vskip .5ex
    \vskip -.125in
    {\answer{\color{red}}\question{\color{.}}
     \vskip 0.5ex \hrule height 0.4pt \vskip 0.5ex}
  }}
}

% -----------------------------------------------------------------------------
% macros
% -----------------------------------------------------------------------------

% abbreviations ---------------------------------------------------------------

\def \<{\langle}
\def \>{\rangle}

\def \ε{\varepisilon}
\def \θ{\vartheta}
\def \κ{\varkappa}
\def \π{\varpi}
\def \ρ{\varrho}
\def \σ{\varsigma}
\def \φ{\varphi}

\def \Γ{\varGamma}
\def \Δ{\varDelta}
\def \Θ{\varTheta}
\def \Λ{\varLambda}
\def \Ξ{\varXi}
\def \Π{\varPi}
\def \Σ{\varSigma}
\def \Υ{\varUpsilon}
\def \Φ{\varPhi}
\def \Ψ{\varPsi}
\def \Ω{\varOmega}

% special characters ----------------------------------------------------------

\catcode `α = \active \let α \alpha
\catcode `β = \active \let β \beta
\catcode `γ = \active \let γ \gamma
\catcode `δ = \active \let δ \delta
\catcode `ε = \active \let ε \epsilon
\catcode `ζ = \active \let ζ \zeta
\catcode `η = \active \let η \eta
\catcode `θ = \active \let θ \theta
\catcode `ι = \active \let ι \iota
\catcode `κ = \active \let κ \kappa
\catcode `λ = \active \let λ \lambda
\catcode `μ = \active \let μ \mu
\catcode `ν = \active \let ν \nu
\catcode `ξ = \active \let ξ \xi
\catcode `ο = \active \let ο o
\catcode `π = \active \let π \pi
\catcode `ρ = \active \let ρ \rho
\catcode `σ = \active \let σ \sigma
\catcode `τ = \active \let τ \tau
\catcode `υ = \active \let υ \upsilon
\catcode `φ = \active \let φ \phi
\catcode `χ = \active \let χ \chi
\catcode `ψ = \active \let ψ \psi
\catcode `ω = \active \let ω \omega

\catcode `Α = \active \let Α A
\catcode `Β = \active \let Β B
\catcode `Γ = \active \let Γ \Gamma
\catcode `Δ = \active \let Δ \Delta
\catcode `Ε = \active \let Ε E
\catcode `Ζ = \active \let Ζ Z
\catcode `Η = \active \let Η H
\catcode `Θ = \active \let Θ \Theta
\catcode `Ι = \active \let Ι I
\catcode `Κ = \active \let Κ K
\catcode `Λ = \active \let Λ \Lambda
\catcode `Μ = \active \let Μ M
\catcode `Ν = \active \let Ν N
\catcode `Ξ = \active \let Ξ \Xi
\catcode `Ο = \active \let Ο O
\catcode `Π = \active \let Π \Pi
\catcode `Ρ = \active \let Ρ P
\catcode `Σ = \active \let Σ \Sigma
\catcode `Τ = \active \let Τ T
\catcode `Υ = \active \let Υ \Upsilon
\catcode `Φ = \active \let Φ \Phi
\catcode `Χ = \active \let Χ X
\catcode `Ψ = \active \let Ψ \Psi
\catcode `Ω = \active \let Ω \Omega

% other -----------------------------------------------------------------------

\long\def \textAnswer #1{\answer{{\color{red!50!black}#1}}}

\long\def \standardsSection #1{{
  \section*{Standards}
  \def\arraystretch{2}
  \setlength\tabcolsep{1em}
  \begin{tabular}{|r|l|}
    \hline \textbf{Standard} & \textbf{Score} \\\hline #1
  \end{tabular}
}}
\long\def \standards #1{{
  \vspace{-1em}
  \def\arraystretch{2}
  \setlength\tabcolsep{1em}
  \begin{tabular}{|r|l|}
    \hline #1
  \end{tabular}
  \vspace{1em}
}}
\def \standard #1{
  #1 & ~~~ \\\hline
}

% -----------------------------------------------------------------------------
% }}}}}

% -----------------------------------------------------------------------------
% document
% -----------------------------------------------------------------------------

\begin{document}
\thispagestyle{firstpage}

\section*{Instructions}
\begin{itemize}
  \item Fill in your name and CWID on every page.
\end{itemize}

\standardsSection{
  \standard{review :: general knowledge}
  \standard{data types and representation}
  \standard{operators}
  \standard{control flow}
}

\vfill
testing

\newpage

\section{Question}
\standards{
  \standard{review :: general knowledge}
}

Define or describe each of the following
\begin{itemize}
  \item C++ compilation process
    \begin{itemize}
      \item preprocess
        \textAnswer{\\
          \url{http://faculty.cs.niu.edu/~mcmahon/CS241/Notes/compile.html}
        }
        \vfill
      \item compile
        \textAnswer{\\
          \url{http://faculty.cs.niu.edu/~mcmahon/CS241/Notes/compile.html}
        }
        \vfill
      \item assmble
        \textAnswer{\\
          \url{http://faculty.cs.niu.edu/~mcmahon/CS241/Notes/compile.html}
        }
        \vfill
      \item link
        \textAnswer{\\
          \url{http://faculty.cs.niu.edu/~mcmahon/CS241/Notes/compile.html}
        }
        \vfill
    \end{itemize}
  \item program execution process
    \begin{itemize}
      \item load
        \textAnswer{\\
          \url{https://en.wikipedia.org/wiki/Loader_(computing)}
        }
        \vfill
      \item run
        \textAnswer{\\
          \url{https://en.wikipedia.org/wiki/Execution_(computing)}
        }
        \vfill
    \end{itemize}
  \item named constant vs object-like macro
    (e.g.~\mintinline{cpp}{const int ten = 10;} vs
    \mintinline{cpp}{#define TEN 10})
    \textAnswer{\\
      A named constant is like a variable except that its value cannot be
      changed, while an object-like macro is (roughly) a string of code that
      will be replaced by the preprocessor with some other string of code
      before compilation.  This has many implications, depending on how each is
      used, though in many simple cases (with compiler optimization on) the
      generated code is the same.
      \begin{itemize}
        \item \url{http://duramecho.com/ComputerInformation/WhyHowCppConst.html}
        \item \url{https://gcc.gnu.org/onlinedocs/cpp/Object-like-Macros.html#Object-like-Macros}
      \end{itemize}
    }
    \vfill
  \item bit, byte, word
    \textAnswer{\\
      \url{https://en.wikipedia.org/wiki/Bit}
      \url{https://en.wikipedia.org/wiki/Byte}
      \url{https://en.wikipedia.org/wiki/Word_(computer_architecture)}
    }
    \vfill
\end{itemize}
    testing

\end{document}

