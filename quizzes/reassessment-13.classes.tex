% -----------------------------------------------------------------------------
% Copyright &copy; 2015 Ben Blazak <bblazak@fullerton.edu>  
% This work is licensed under a [Creative Commons Attribution 4.0 International
% License] (http://creativecommons.org/licenses/by/4.0/)  
% -----------------------------------------------------------------------------

% -----------------------------------------------------------------------------
% Copyright &copy; 2015 Ben Blazak <bblazak@fullerton.edu>  
% This work is licensed under a [Creative Commons Attribution 4.0 International
% License] (http://creativecommons.org/licenses/by/4.0/)  
% -----------------------------------------------------------------------------

% - [Which lua environment should I use?] (http://tex.stackexchange.com/a/33102)
\directlua{require('scripts/functions.lua')}

\ifx\question\undefined
  \long\def \question #1{#1}
\fi
\ifx\answer\undefined
  \long\def \answer #1{#1}
\fi

% -----------------------------------------------------------------------------

\def \docTitle  {\directlua{doc_title('\jobname')}}
\def \docAuthor {\question{Name:\hfill{}CWID:\hfill}\answer{Answer Key}}
\def \docClass  {CPSC 121-11}
\def \docSchool {~}
\def \docTerm   {CSUF Fall 2015}

\def \docCopyright {
  \begin{figure}[b!]
    \tiny
    {\answer{\color{red}}\question{\color{.}}\hrule}
    \vspace{2ex}
    \begin{minipage}[c]{0.10\textwidth} \vspace{0ex} \vspace{-1pt}
      \includegraphics[height=5ex]{images/ccby}
    \end{minipage}
    \begin{minipage}[c]{0.90\textwidth} \vspace{0ex}
      Copyright \copyright{} Ben Blazak \url{bblazak@fullerton.edu}\\
      This work is licensed under a Creative Commons Attribution 4.0
      International License \url{http://creativecommons.org/licenses/by/4.0/}
    \end{minipage}
    \vspace{-.14in}
    \vspace{2ex}
  \end{figure}
}

% -----------------------------------------------------------------------------

\documentclass{article}

\usepackage[includehead,
%             includefoot,
            margin=1in,
            top=.25in,
            headheight=.75in,
            headsep=.25in,
            footskip=.25in,
           ]{geometry}

\usepackage[fleqn]{amsmath}
\usepackage{amssymb}
\usepackage[shortlabels]{enumitem}
\usepackage{fancyhdr}
\usepackage{lastpage}
\usepackage{minted}
\usepackage{tikz}
\usepackage[explicit]{titlesec}

\usepackage[T1]{fontenc}
\usepackage[utf8]{inputenc}
\usepackage{lmodern}

\usepackage{xcolor}
\usepackage[framemethod=TikZ]{mdframed}

% - [Illegal parameter number in definition of \Hy@tempa]
%   (https://kba49.wordpress.com/2013/04/12/illegal-parameter-number-in-definition-of-hytempa/)
\usepackage{hyperref}

% text ------------------------------------------------------------------------

\binoppenalty = 10000  % never break next to a binary operator
\relpenalty   = 10000  % never break next to a relation operator

\setlength{\parindent}{0em}
\setlength{\parskip}{1ex}

\setlist[itemize]{nosep,itemsep=.5ex,parsep=.5ex}

% sections --------------------------------------------------------------------

% - <http://tex.stackexchange.com/a/58299>
\titleformat{\section}{\normalfont\Large\bfseries}{}{0em}{#1\ \thesection}

% math ------------------------------------------------------------------------

\setlength{\mathindent}{1cm}

% - "\begin{document}" resets these values, so they have to be treated
%   specially
\AtBeginDocument{
  \setlength{\abovedisplayskip}{1.5ex plus .5ex minus .5ex}
  \setlength{\belowdisplayskip}{1.5ex plus .5ex minus .5ex}
}

% source code -----------------------------------------------------------------

\usemintedstyle{solarizedlight}

\fvset{samepage=true}

% header and footer -----------------------------------------------------------

\pagestyle{fancy}

\renewcommand{\headrule}{{\answer{\color{red}}\question{\color{.}}
                          \vskip 0.5ex \hrule height 0.4pt \vskip 0.5ex}}
\renewcommand{\footrule}{{\answer{\color{red}}\question{\color{.}}
                          \vskip 0.5ex \hrule height 0.4pt \vskip 0.5ex}}

\fancyhead[L]{{\answer{\color{red}}\question{\color{.}}\docAuthor}
              {\color{.}\\\docClass}}
\fancyhead[C]{{\color{.}\docSchool\\}}
\fancyhead[R]{{\color{.}\docTerm\\\docTitle}}

\fancyfoot[C]{\thepage{} of \pageref{LastPage}}

% -----------------------------------------------------------------------------
% macros
% -----------------------------------------------------------------------------

% abbreviations ---------------------------------------------------------------

\def \<{\langle}
\def \>{\rangle}

\def \ε{\varepisilon}
\def \θ{\vartheta}
\def \κ{\varkappa}
\def \π{\varpi}
\def \ρ{\varrho}
\def \σ{\varsigma}
\def \φ{\varphi}

\def \Γ{\varGamma}
\def \Δ{\varDelta}
\def \Θ{\varTheta}
\def \Λ{\varLambda}
\def \Ξ{\varXi}
\def \Π{\varPi}
\def \Σ{\varSigma}
\def \Υ{\varUpsilon}
\def \Φ{\varPhi}
\def \Ψ{\varPsi}
\def \Ω{\varOmega}

% special characters ----------------------------------------------------------

\catcode `α = \active \let α \alpha
\catcode `β = \active \let β \beta
\catcode `γ = \active \let γ \gamma
\catcode `δ = \active \let δ \delta
\catcode `ε = \active \let ε \epsilon
\catcode `ζ = \active \let ζ \zeta
\catcode `η = \active \let η \eta
\catcode `θ = \active \let θ \theta
\catcode `ι = \active \let ι \iota
\catcode `κ = \active \let κ \kappa
\catcode `λ = \active \let λ \lambda
\catcode `μ = \active \let μ \mu
\catcode `ν = \active \let ν \nu
\catcode `ξ = \active \let ξ \xi
\catcode `ο = \active \let ο o
\catcode `π = \active \let π \pi
\catcode `ρ = \active \let ρ \rho
\catcode `σ = \active \let σ \sigma
\catcode `τ = \active \let τ \tau
\catcode `υ = \active \let υ \upsilon
\catcode `φ = \active \let φ \phi
\catcode `χ = \active \let χ \chi
\catcode `ψ = \active \let ψ \psi
\catcode `ω = \active \let ω \omega

\catcode `Α = \active \let Α A
\catcode `Β = \active \let Β B
\catcode `Γ = \active \let Γ \Gamma
\catcode `Δ = \active \let Δ \Delta
\catcode `Ε = \active \let Ε E
\catcode `Ζ = \active \let Ζ Z
\catcode `Η = \active \let Η H
\catcode `Θ = \active \let Θ \Theta
\catcode `Ι = \active \let Ι I
\catcode `Κ = \active \let Κ K
\catcode `Λ = \active \let Λ \Lambda
\catcode `Μ = \active \let Μ M
\catcode `Ν = \active \let Ν N
\catcode `Ξ = \active \let Ξ \Xi
\catcode `Ο = \active \let Ο O
\catcode `Π = \active \let Π \Pi
\catcode `Ρ = \active \let Ρ P
\catcode `Σ = \active \let Σ \Sigma
\catcode `Τ = \active \let Τ T
\catcode `Υ = \active \let Υ \Upsilon
\catcode `Φ = \active \let Φ \Phi
\catcode `Χ = \active \let Χ X
\catcode `Ψ = \active \let Ψ \Psi
\catcode `Ω = \active \let Ω \Omega

% other -----------------------------------------------------------------------

\long\def \textAnswer #1{\answer{{\color{red!50!black}#1}}}
\long\def \textQuestion #1{\question{{\color{.}#1}}}

\long\def \standardsSection #1{{
  \section*{Standards}
  \def\arraystretch{2}
  \setlength\tabcolsep{1em}
  \begin{tabular}{|r|l|}
    \hline \textbf{Standard} & \textbf{Score} \\\hline #1
  \end{tabular}
  \docCopyright
  \newpage
}
  \def\docCopyright{}
}
\long\def \standards #1{{
  \vspace{-1em}
  \def\arraystretch{2}
  \setlength\tabcolsep{1em}
  \begin{tabular}{|r|l|}
    \hline #1
  \end{tabular}
  \vspace{1em}
}}
\def \standard #1{
  #1 & ~~~ \\\hline
}

\def \program #1{{
  \def \lCodedir{\directlua{doc_name('\jobname')}}
  \def \lCode{./code/\lCodedir/#1.cpp}
  \def \lQuestion{./code/\lCodedir/#1.gen.question.txt}
  \def \lAnswer{./code/\lCodedir/#1.gen.answer.txt}
  %
  \inputminted[frame=single]{cpp}{\lCode}
  \textQuestion{
    \inputminted[frame=single,
                 label={\normalsize{}Output},
                 fontsize=\Large,
                ]{text}{\lQuestion}
  }
  \textAnswer{
    \inputminted[frame=single,
                 label={Output},
                ]{text}{\lAnswer}
  }
}}



% -----------------------------------------------------------------------------

\begin{document}
\docCopyright

\standardsSection{
  \standard{classes}
}

\newEvenPage

\section{Question}
\standards{
  \standard{classes}
}

Before you begin, please
\begin{itemize}
  \item Remember to preface your program with a block comment containing a
    short description of what it does.
  \item Be sure to write all necessary (and no unnecessary)
    \mintinline{cpp}{#include} directives and \mintinline{cpp}{using}
    statements.
  \item Remember that accessors (i.e.~getter methods) should always be
    \mintinline{cpp}{const}.
  \item Try to comment with the level of detail you would find helpful (but not
    irritating) in code of similar complexity written by another student in
    this class.
\end{itemize}

Write a \mintinline{cpp}{CountrySong} class with the following
\mintinline{cpp}{private} members, all of type \mintinline{cpp}{bool}:
\begin{itemize}
  \item \mintinline{cpp}{girl}
  \item \mintinline{cpp}{truck}
  \item \mintinline{cpp}{dog}
\end{itemize}

Write getters and setters, with inline implementations, for each of the three
\mintinline{cpp}{private} data members.  In at least one of the setters, accept
an argument with the same name as the \mintinline{cpp}{private} data member
being set, and use the \mintinline{cpp}{this->} syntax to change the value of
the data member.

Prototype a constructor for the class capable of taking 0, 1, 2, or 3 arguments
(i.e.~a constructor with three arguments, each of which has a default value).
Let the default value of each argument be \mintinline{cpp}{true}.

Prototype a method named \mintinline{cpp}{sing} that is
\mintinline{cpp}{const}, accepts no arguments, and returns
\mintinline{cpp}{void}.

Outside the class, define the constructor that was prototyped above.  Use the
initialization list to set the values of all data members.  The body of this
function should be empty.

Below the definition of the constructor, define the \mintinline{cpp}{sing}
method to do the following:
\begin{itemize}
  \item If \mintinline{cpp}{girl} is \mintinline{cpp}{true} output
    \mintinline{cpp}{"My girl, she left me\n"}
  \item If \mintinline{cpp}{truck} is \mintinline{cpp}{true} output
    \mintinline{cpp}{"My pickup broke down\n"}
  \item If \mintinline{cpp}{dog} is \mintinline{cpp}{true} output
    \mintinline{cpp}{"My dog, he died\n"}
  \item Finally, output \mintinline{cpp}{"In this small small town\n"}
\end{itemize}

Write a short \mintinline{cpp}{main()} to do the following:
\begin{itemize}
  \item Declare and initialize (without using the \mintinline{cpp}{=}
    operator) a \mintinline{cpp}{CountrySong} named \mintinline{cpp}{a} such
    that all three private data members are set to \mintinline{cpp}{true}.
  \item Declare and initialize (using the \mintinline{cpp}{=} operator) a
    \mintinline{cpp}{CountrySong} named \mintinline{cpp}{b} such that only the
    \mintinline{cpp}{truck} data member is set to \mintinline{cpp}{true}.
  \item Declare a \mintinline{cpp}{CountrySong} named \mintinline{cpp}{c} and
    initialize it by setting it equal to the \mintinline{cpp}{CountrySong}
    named \mintinline{cpp}{a}.
  \item Use the setter for the \mintinline{cpp}{truck} data member to set
    \mintinline{cpp}{truck} to \mintinline{cpp}{false} in the
    \mintinline{cpp}{CountrySong} named \mintinline{cpp}{c}.
  \item Call the \mintinline{cpp}{sing} method on \mintinline{cpp}{a},
    \mintinline{cpp}{b}, and \mintinline{cpp}{c}, in that order (you need not
    note the output in this case, but you may if you wish).
\end{itemize}

\newpage

\textQuestion{\makePageQuadrilleRuled}
\textAnswer{\inputminted{cpp}{\docCodeDir/.classes.cpp.gen.section.all}}

\newOddPage

\textQuestion{\makePageQuadrilleRuled}
\textAnswer{\inputminted[label=Output]{text}{\docCodeDir/.classes.gen.output}}

\end{document}

